% Reportes técnicos --------------------------------------------------------------------------------

% Carga de paquetes --------------------------------------------------------------------------------
\usepackage{booktabs}
\usepackage{makecell}
\usepackage{float,lscape}
\usepackage[explicit]{titlesec}
\usepackage{tocbibind}
\usepackage{ragged2e}
\usepackage[spanish,es-tabla]{babel}            
\usepackage[utf8]{inputenc}
\usepackage[T1]{fontenc} % Libertine
\usepackage{libertine} % Libertine
\usepackage[autostyle,spanish=mexican]{csquotes}
% \usepackage{tgtermes} % Times New Roman
% \usepackage{tgpagella} % Pagella
% \usepackage{lmodern} % Computer modern

% enumerados en listas
\usepackage[shortlabels]{enumitem}

\usepackage{pdflscape}	
\usepackage{afterpage}	
\usepackage{dirtree}
\usepackage{graphicx}
\usepackage[top=2cm, bottom=3cm, left=3cm, right=2cm, includeheadfoot]{geometry}

\usepackage{dsfont}
\usepackage{mathtools}
\usepackage{mathrsfs}
\usepackage[none]{hyphenat}

% AMS Math + AMS symbols
\usepackage{amsmath}                   
\usepackage{amssymb}        
\usepackage{amsthm}
% \usepackage{amsfonts}
\usepackage[cal=dutchcal]{mathalfa}
\usepackage[figuresright]{rotating}

% Sideways of figures & tables 
\usepackage{setspace}
\usepackage{booktabs}

% Símbolos especiales para textos de computacion
\usepackage{url}
\usepackage[gen]{eurosym}
\usepackage[below]{placeins}

% Símbolos actuariales
\usepackage{actuarialangle}
\usepackage{actuarialsymbol}

% Símbolos matemáticos
\usepackage{bbm}

% Text Companion fonts
\usepackage{textcomp}

% Escribir simbolos especiales Ej: o/oo
\usepackage{gensymb}

% Escribir simbolos especiales Ej: o/oo
\usepackage{enumitem} 

% Para personalizar los labels de items en listas
\usepackage{multirow}

% \usepackage[belowskip=-10pt,aboveskip=5pt]{caption}
\usepackage[belowskip=0pt,aboveskip=0pt]{caption}

\usepackage{draftwatermark} % para poner watermark
%\usepackage[some]{background} % para imágenes como marca de agua en ciertas páginas
\usepackage[
pages=some,
contents={},
opacity=1,
scale=1.5,
color=blue!90
]{background}
\usepackage{ifthen}

% Colocar figuras en la posición correcta
\usepackage{float}

% Dar formato a fechas
\usepackage{datetime}

% Hypervínculos en PDF
\usepackage[colorlinks = true, linkcolor = iess_blue, urlcolor = iess_blue, citecolor = iess_green, 
  linktoc = page]{hyperref}
\usepackage{memhfixc}

% Incluir docuemntos en formato pdf
\usepackage{pdfpages}
\usepackage{lastpage}

% Bibliografía
\usepackage[style=numeric-verb, bibstyle=numeric, citestyle=numeric, backend=biber, natbib=true, 
  maxnames=10, maxcitenames=2]{biblatex}

\usepackage{tikz}
\usetikzlibrary{cd,babel,automata,matrix,positioning,decorations.pathreplacing,arrows}

% Tablas grandes------------------------------------------------------------------------------------
\usepackage{longtable}
\usepackage{ctable}
\usepackage{color}
\usepackage{colortbl}
\usepackage{titleps}

% % Píe de notas en las tablas
% \usepackage{threeparttable}

%Necesario para paquete stargazer-------------------------------------------------------------------
\usepackage{dcolumn}

\usepackage{framed, xcolor}

\usepackage{icomma}

\definecolor{shadecolor}{rgb}{0.91, 0.84, 0.42}

\spanishdecimal{,}

%Environment Definición-----------------------------------------------------------------------------

\newtheorem{definition}{Definición}[section]

% Citas legales ------------------------------------------------------------------------------------
\newbibmacro{legalcite}{
  \textit{\printfield{shorttitle}}\,
  [\printtext[bibhyperref]{\printfield{labelnumber}}]
}

\DeclareCiteCommand{\legalcite}
  {\usebibmacro{prenote}}
  {\usebibmacro{citeindex}
   \usebibmacro{legalcite}}
  {\multicitedelim}
  {\usebibmacro{postnote}
}

\newbibmacro{estcite}{
  \textit{\printfield{shorttitle}}\,
  [\printtext[bibhyperref]{\printfield{labelnumber}}]
}
  
\DeclareCiteCommand{\estcite}
  {\usebibmacro{prenote}}
  {\usebibmacro{citeindex}
   \usebibmacro{estcite}}
  {\multicitedelim}
  {\usebibmacro{postnote}
}

% Colours ------------------------------------------------------------------------------------------
\definecolor{iess_blue}{RGB}{ 0, 63, 138 }
\definecolor{iess_green}{RGB}{ 0, 116, 53 }
\definecolor{cover_blue}{RGB}{ 216, 246, 255 }
\definecolor{Azul}{rgb}{0.00000, 0.265625, 0.578125}
\definecolor{Gris}{rgb}{0.85, 0.85, 0.85}

\definecolor{iess_green1}{RGB}{ 0, 115, 53 }
\definecolor{iess_green2}{RGB}{ 0, 114, 53 }
\definecolor{iess_green3}{RGB}{ 0, 113, 53 }
% Page style ---------------------------------------------------------------------------------------
\setlength{\headsep}{1.5cm}

\newpagestyle{repstyle}{
\setfoot[\fontsize{8pt}{6pt}\thepage\hspace{1pt} de \pageref*{LastPage}][][]
{}{}{\fontsize{8pt}{6pt}\thepage\hspace{1pt} de \pageref*{LastPage}}
\sethead[\scriptsize{ \thechapter. \chaptertitle } ]
[]
[\scriptsize{\includegraphics[width=.022\textwidth]{graficos/logo_iess_azul.png} \hspace{0.1cm}
}]
{\scriptsize{\includegraphics[width=.022\textwidth]{graficos/logo_iess_azul.png} \hspace{0.1cm}
}}
{}
{\scriptsize{ \thechapter. \chaptertitle } }

\setheadrule{0.4pt}
}

% Pies de pagina sin marcador
\newcommand\blfootnote[1]{%
  \begingroup
  \renewcommand\thefootnote{}\footnote{#1}%
  \addtocounter{footnote}{-1}%
  \endgroup
}

\newpagestyle{repheadstyle}{
\setfoot[\fontsize{8pt}{6pt}\thepage\hspace{1pt} de \pageref*{LastPage}][][]
{}{}{\fontsize{8pt}{6pt}\thepage\hspace{1pt} de \pageref*{LastPage}}
\sethead[\scriptsize{\chaptertitle}]
[]
[\scriptsize{\includegraphics[width=.022\textwidth]{graficos/logo_iess_azul.png} \hspace{0.1cm}
}]
{\scriptsize{\includegraphics[width=.022\textwidth]{graficos/logo_iess_azul.png} \hspace{0.1cm}
}}
{}
{\scriptsize{\chaptertitle}}
\setheadrule{0.4pt}
}

\newpagestyle{repchapstyle}{
\setfoot[\fontsize{8pt}{6pt}\thepage\hspace{1pt} de \pageref*{LastPage}][][]
{}{}{\fontsize{8pt}{6pt}\thepage\hspace{1pt} de \pageref*{LastPage}}
\sethead[][][]{}{}{}
}

\assignpagestyle{\chapter}{repchapstyle}

% Index generation ---------------------------------------------------------------------------------
\makeindex


% Estilo para capítulos y secciones ----------------------------------------------------------------
\titleformat{\chapter}
{\normalfont}
{}
{0cm}
{\filright\fontsize{22}{22}\bfseries\selectfont{\textcolor{iess_green}{\thechapter\hspace{4mm}#1}}} 
\titlespacing{\chapter}{0.0cm}{0.0\baselineskip}{3.0\baselineskip}

\titleformat{name = \chapter, numberless}
{\normalfont}
{}
{0cm}
{\filright\fontsize{22}{22}\bfseries\selectfont{\textcolor{iess_green}{#1}}}
\titlespacing{\chapter}{0.0cm}{0.0\baselineskip}{3.0\baselineskip}

\titleformat{\section}
{\normalfont}
{}
{0cm}
{\filright\fontsize{16}{21}\bfseries\selectfont{\textcolor{iess_green}{\thesection\hspace{4mm}#1}}}
\titlespacing{\section}{0.0cm}{0.0\baselineskip}{0.0\baselineskip}

\titleformat{\subsection}
{\normalfont}
{}
{0cm}
{\filright\fontsize{14}{18}\bfseries\selectfont{\textcolor{iess_green}{\thesubsection\hspace{4mm}#1}}}
\titlespacing{\subsection}{0.0cm}{0.0\baselineskip}{0.0\baselineskip}

\titleformat{\subsubsection}
{\normalfont}
{}
{0cm}
{\fontsize{12}{16}\bfseries\selectfont{\textcolor{iess_green}{\thesubsubsection\hspace{4mm}#1}}}
\titlespacing{\subsubsection}{0.0cm}{0.0\baselineskip}{0.0\baselineskip}


\titleformat{\paragraph}
{\normalfont}
{}
{0cm}
{\fontsize{12}{16}\bfseries\selectfont{\textcolor{iess_green}{\theparagraph\hspace{4mm}#1}}}
\titlespacing{\paragraph}{0.0cm}{0.0\baselineskip}{0.0\baselineskip}

\setcounter{secnumdepth}{4} % nivel de númeración hasta 4

% \titleformat{\paragraph}
% {\normalfont\normalsize\bfseries}{\theparagraph}{1em}{}
% \titlespacing*{\paragraph}
% {0pt}{3.25ex plus 1ex minus .2ex}{1.5ex plus .2ex}

% Table of contents style --------------------------------------------------------------------------
% \settocdepth{subsection}
% \setsecnumdepth{subsection}
% \maxsecnumdepth{subsection}
% \maxtocdepth{subsection}

% Commands operators -------------------------------------------------------------------------------
\newcommand{\expect}{\operatorname{\mathbb{E}}\expectarg}
	\DeclarePairedDelimiterX{\expectarg}[1]{[}{]}{%
		\ifnum\currentgrouptype=16 \else\begingroup\fi
		\activatebar#1
		\ifnum\currentgrouptype=16 \else\endgroup\fi
}

\newcommand{\var}{\operatorname{\mathbb{V}}\vararg}
\DeclarePairedDelimiterX{\vararg}[1]{[}{]}{%
	\ifnum\currentgrouptype=16 \else\begingroup\fi
	\activatebar#1
	\ifnum\currentgrouptype=16 \else\endgroup\fi
}

\newcommand{\cov}{\operatorname{Cov}\covarg}
\DeclarePairedDelimiterX{\covarg}[1]{[}{]}{%
	\ifnum\currentgrouptype=16 \else\begingroup\fi
	\activatebar#1
	\ifnum\currentgrouptype=16 \else\endgroup\fibb
}

\addto\captionsspanish{\renewcommand*\contentsname{Contenidos}}
\addto\captionsspanish{\renewcommand*\listfigurename{Figuras}}
\addto\captionsspanish{\renewcommand*\listtablename{Tablas}}

% \renewcommand{\appendixpagename}{\sffamily\bfseries{Anexos}}
% \renewcommand{\appendixtocname}{Anexos}
\renewcommand{\appendixname}{Anexo}
\renewcommand{\chaptername}{Capítulo}
\renewcommand{\tablename}{Tabla}
\renewcommand{\labelitemi}{$\bullet$}
\renewcommand{\labelitemii}{$\cdot$}

\DeclareMathOperator{\Liminf}{lim\,inf\,}
\DeclareMathOperator{\sign}{sign}
\DeclareMathOperator{\argmin}{arg\,min\,}
\DeclareMathOperator{\arginf}{arg\,inf\,}

\thinmuskip=3mu plus 0mu minus 0mu
\medmuskip=4mu plus 0mu minus 0mu
\thickmuskip=5mu plus 0mu minus 0mu

\newcolumntype{L}[1]{>{\raggedright\arraybackslash}p{#1}}
\newcolumntype{C}[1]{>{\centering\arraybackslash}p{#1}}
\newcolumntype{R}[1]{>{\raggedleft\arraybackslash}p{#1}}

% Otros --------------------------------------------------------------------------------------------
% \linespread{1.5\baselineskip} %interlineado a espacio y medio
% \linespread{2.0\baselineskip} %interlineado a doble espacio
\setlength{\parskip}{1.0\baselineskip} % espacio entre parrafos
\setlength{\parindent}{0mm} % indentacion

% \setlength{\cftsecnumwidth}{1.0\baselineskip}
% \setlength{\cftsubsecnumwidth}{1.0\baselineskip}

% New environements --------------------------------------------------------------------------------
\theoremstyle{plain} 
\newtheorem{ejemplo}{Ejemplo}
\newtheorem{observacion}{Observación}[section]
\newtheorem{comentario}{Comentario}[section]
\newtheorem{nota}{Nota}[section]

% Longtable configuration --------------------------------------------------------------------------
% Espacio entre las tablas o graficos y el texto
% \setlength{\textfloatsep}{\baselineskip plus 0.2\baselineskip minus 0.2\baselineskip}
% \setlength{\intextsep}{\baselineskip plus 0.2\baselineskip minus 0.2\baselineskip}
\setlength{\floatsep}{5pt}
\setlength{\textfloatsep}{\baselineskip}
\setlength{\intextsep}{\baselineskip}
\setlength{\abovedisplayskip}{0.25\baselineskip}
\setlength{\belowdisplayskip}{0.25\baselineskip}
\setlength{\abovecaptionskip}{0.2\baselineskip}
\setlength{\belowcaptionskip}{0.0\baselineskip}
\setlength{\LTpre}{\baselineskip}
\setlength{\LTpost}{0\baselineskip}

% Caption configuration ----------------------------------------------------------------------------
\captionsetup[figure]{format=hang,justification=RaggedRight,singlelinecheck=true,width=.8\textwidth,
font=small,position=top,belowskip=0pt,aboveskip=0pt}
\captionsetup[table]{format=hang,justification=RaggedRight,singlelinecheck=true,width=.8\textwidth,
font=small,position=top,belowskip=0pt,aboveskip=0pt}

%\renewcommand{\mkbegdispquote}[2]{\itshape}
%\renewcommand{\mktextquote}[6]{#1\textit{#2}#4#3#6#5}

\makeatletter
\AddToHook{env/tabular/begin}{\let\input\@@input}
\makeatother

\sloppy

